\documentclass[uplatex,a4paper,11pt,dvipdfmxs]{jsarticle}
\usepackage[dvipdfmx]{graphicx}
\usepackage{amsmath}
\usepackage{float}
\begin{document}
\begin{enumerate}
    \item {\bf \large 実験目的}\\
    \normalsize \quad オペアンプ単体で使用した時の基本的な性質について学ぶ. 次にオペアンプ回路の中でも簡単な例である
    ボルテージ・フォロワを使って, 負帰還回路の性質について学ぶ.\\
    
    \item {\bf \large 原理}\\
    \quad オペアンプとは, operational Amplifierの略で, 演算増幅器という意味がある. オペアンプを使った回路では加算, 減算, 微積分
    などの数学的な演算や増幅の機能がある. 
    \begin{figure}[H]
    \centering
    \includegraphics[width=9cm]{/home/asakayusuke/Pictures/z31.png}
    \caption{オペアンプについて}
    \label{figure:op}
    \end{figure}
    \quad 最初にオペアンプの基本的な性質について説明しよう. オペアンプは図\ref{figure:op}のような三角形の記
    号を使って表される. オペアンプを使用するときにはまずプラスの直流電源+V\textsubscript{CC}と
    マイナスの直流電源\(-V\textsubscript{CC}\)に接続し, 電力を供給する必要がある. オペアンプは二つ
    の入力端子と一つの出力端子を持つ. 二つの入力端子のうち一つは非反転入力端子(図
    中の+端子)と呼ばれる. ここから入力された電圧\(V_+\)は, 符号はそのまま (非反転)オペ
    アンプに入力される. もう一つは反転入力端子(図中の\(-\)端子)と呼ばれる. ここから
    入力された電圧\(V_-\)は、オペアンプ内部で符号が反転される。オペアンプの非反転入力端
    子 (+端子)に入力電圧\(V_+\), 反転入力端子 (\(-\)端子)に入力電圧\(V_-\)が同時に入力されたとき, オペアンプの出力端子からは\\
    \begin{equation}
        \(
        V_o=A(V_+-V_-) \label{eq:vo}
        \)
    \end{equation}
    の電圧が出力される. つまり入力電圧\(V_+\)と入力電圧\(V_-\)の差分\((V_+-V_-)\)がA倍に増幅された電圧\(V_o\)が出力される. 
    Aのことを開ループゲイン (オープンループゲイン)という. 理想的なオペアンプではAは\(\infty \) (無限大)の大きさになる. つまり入力電圧
    の差分\((V_+-V_-)\)が0より少しでも大きければ電圧\(V_o\)は+\(\infty \) となり, 一方\((V_+-V_-)\)が0より少しでも小さけ
    れば\(V_o\)は\(-\infty \) となる. しかしながら実際のオペアンプの出力電圧は最大でもオペアンプ
    を駆動している直流電源の電圧\(\pm V\textsubscript{CC}\)に制限される. 従って増幅率は\(\infty \) ではない. その他のオペアンプ
    の重要な性質として, 入力インピーダンスが高く, 出力インピーダンスが低いという性質がある. \\
    \quad 図\ref{figure:op}のようなオペアンプを単体で使用する例としてコンパレータ回路があげられる. しか
    しながらオペアンプを単体で使用することはまれであり, また非常に扱いにくい. 実際
    に使用されるオペアンプ回路では出力電圧\(V_o\)の一部あるいは全部を入力部に戻しなが
    ら使用する場合がほとんどである. このような出力を入力に戻す操作を帰還 (Feed-back)と呼ぶ. 出力電圧\(V_o\)をオペアンプの反転入力
    端子 (\(-\)端子)に帰還する場合を負帰還 (Negative feed-back), 非反転入力端子 (+端子)に帰還する場合を正帰還 (Positivefeed-back)と呼ぶ. 
    実際のオペアンプ回路では負帰還を利用したものがほとんどであり, その例として後の章で学ぶ増幅回路やフィルター回路などがあげられる. 
    本実験では負帰還を用いたオペアンプ回路の中でも最も単純なボルテージ・フォロワ回路を扱う. 
    一方, オペアンプを正帰還で利用する代表的な例は発振回路である. \\
    \quad 次に, コンパレータ回路について説明する. コンパレータとは何か基準とな
    る電圧に対して, 入力電圧がそれよりも高いか低いかを判定する場合に使用する. 実験1の図\ref{figure:i}
    では反転入力端子 (\(-\)端子)への入力電圧\(V_-\)を基準電圧とする. 反転入力端子 (\(-\)端子)は接
    地されているので, 基準電圧は 0[V]となる. つまり\(V_-\)=0[V]である. 一方, 非反転入力
    端子 (+端子)への入力電圧\(V_+\)を入力電圧とする. 仮に入力電圧\(V_+\)が基準電圧\(V_-\)より高い
    場合, コンパレータ回路の出力電圧\(V_o\)は+V\textsubscript{CC}となる. 一方, 入力電圧\(V_+\)が基準電圧\(V_-\)よ
    り低い場合, コンパレータ回路の出力電圧\(V_o\)は\(-V\textsubscript{CC}\)となる. いずれの場合もオペアンプ
    は式 (\ref{eq:vo})で示されたような動作をしているだけである. つまり入力電圧\(V_+\)と基準電圧\(V_-\)
    との差分\((V_+-V_-)\)を非常に高い増幅率Aで増幅しているだけである. しかしながら実際
    のオペアンプは無限大の電圧を出力することができないので, 最大(最小)値である直流電源の
    電圧値\(\pm V\textsubscript{CC}\)を出力することになる. \\
    \quad 最後に, ボルテージ・フォロワ回路について説明する. 実験 1 のコンパレータ回路のようにオペアンプを単体で使用することはまれである. 
    実際にオペアンプを使用する場合, 出力電圧\(V_o\)の一部あるいは全部を反転入力端子 (-端子)にもどして使用する場合がほとんどである. 
    このような回路を負帰還と呼ぶ. 負帰還を含むオペアンプ回路の最も簡単な例は, 図 5 に示すボルテージ・フォロワ回路で
    ある. ボルテージ・フォロワ回路では非反転入力端子 (+端子)に入力電圧\(V_i\)が入力され
    る. またオペアンプからの出力端子と反転入力端子 (\(-\)端子)が導線で直結されており, 
    出力電圧\(V_o\)が帰還率\(\beta =1\)で反転入力端子 (\(-\)端子)に負帰還される. 詳細は【補足説明】
    で説明するが, このような帰還回路が作られているとき, 入力電圧\(V_i\)の値がそのまま
    出力電圧\(V_o\)となって出力され, \(V_o=V_i\)となる. 見方を変えるとボルテージ・フォロワ回路は電圧増幅率\(A_v\)が1の増幅回路となる. \\
    \quad ボルテージ・フォロワ回路についてもう少し説明すると, ボルテージ・フォロワ回路の特徴のひとつは電圧
    増幅率\(A_v\)が1であるということである. 電圧増幅率\(A_v\)が1ということは, 入力電圧\(V_i\)
    の値を変えずにそのまま出力電圧\(V_o\)として出力させるということである. つまり\(V_o=V_i\)
    が成り立つ. 出力電圧\(V_o\)が入力電圧\(V_i\)に追随してフォローするという現象が, ボルテ
    ージ・フォロワ― (voltage follower)という命名の由来である. しかしながらそのよう
    な回路は一見すると意味のない回路であり, 何もしていないように見える. そのような
    回路の役割とはいったい何であろうか?実はボルテージ・フォロワ回路には入力インピ
    ーダンスが非常に大きいという性質がある. 入力インピーダンスが大きいということは, 
    ボルテージ・フォロワ回路に何かしらの信号電圧が入力された時でも, 回路に電流が流
    れないということである. その結果, ボルテージ・フォロワ回路は入力された信号電圧
    を降下させることなく, そのまま正確な電圧値を保ったまま出力側に送ることができる. 
    同時にボルテージ・フォロワ回路は出力インピーダンスが非常に小さいという性質もあ
    る. これは回路の出力先にどのような低い抵抗値を持った負荷が接続されても, 出力電
    圧を低下させることなく, 負荷に十分な電流を流すことができるということである. も
    ちろんオペアンプが持つ最大定格電流値を超えない範囲での話である. \\

    {\bf \large 【補足説明】ボルテージ・フォロワ回路における負帰還について}\\
    \normalsize 負帰還がかかったボルテージ・フォロワ回路における入出力電圧の関係について補足
    説明する. ボルテージ・フォロワ回路では非反転入力端子 (+端子)には外部から入力電
    圧\(V_i\)が入力されている. また反転入力端子 (\(-\)端子)にはオペアンプの出力端子が直結さ
    れており, 出力電圧\(V_o\)が帰還率\(\beta =1(100\%)\)の割合で帰還されている. この場合オペア
    ンプの非反転入力端子 (+), 反転入力端子\((-)\)における入力電圧\(V_+\), \(V_-\)はそれぞれ
    \begin{equation}
        \(
        V_+=V_i\hspace{3},\quad V_-=V_o \label{v2}
        \)
    \end{equation}
    と記述される. ところでオペアンプ自体が実際に行っていることは式で記述される
    ように非反転入力端子 (+端子)への入力電圧\(V_+\)と反転入力端子 (\(-\)端子)への入力電圧\(V_-\)
    の差分\((V_+-V_-)\)をA倍に増幅することだけである. 従ってオペアンプからの出力電圧\(V_o\)
    は (\ref{eq:vo})式に (\ref{v2})式を代入して
    \begin{equation}
        \(
        V_o=A(V_+-V_-)=A(V_i-V_o)
        \)
    \end{equation}
    となる. これを変形すると
    \begin{equation*}
        \(
        (1+A)V_o=AV_i
        \)
    \end{equation*}
    \begin{equation}
        \(
        V_o+\frac{V_o}{A}=V_i \label{v4}
        \)
    \end{equation}
    となる. 通常オペアンプでは開ループでの電圧増幅率 A は非常に大きいため (\ref{v4})式の左
    辺の第 2 項はほぼゼロとなる. よって (\ref{v4})式は
    \begin{equation}
        \(
        V_o=V_i \label{v5}
        \)
    \end{equation}
    となる. これがボルテージ・フォロワにおける入力電圧\(V_i\)と出力電圧\(V_o\)の間の関係で
    ある. つまり出力電圧\(V_o\)と入力電圧\(V_i\)が等しい. 言い換えると電圧増幅率\(A_v\)が1の増幅回路となる. 今のことを数式にすると
    \begin{equation}
        \(
            A_v=\frac{V_o}{V_i}=1
        \)
    \end{equation}
    となる. また (\ref{v5})式の入出力関係を (\ref{v2})式に代入すると, 
    \begin{equation}
        \(
        V_+=V_-
        \)
    \end{equation}
    という関係が得られる. つまり負帰還がかかったオペアンプ回路では非反転入力端子 (+端子)の電圧\(V_+\)と反転入力端子 (\(-\)端子)の電圧
    \(V_-\)は常に等しい. これは非反転入力端子 (+端子)と反転入力端子 (\(-\)端子)があたかも短絡 (ショート)しているかのように振る舞
    うことから仮想短絡 (バーチャル・ショート)と呼ばれる. 負帰還がかかったオペアンプ回路における重要な性質である. \\

    \item {\bf \large 【実験1】コンパレータ回路}\\
    \begin{enumerate}
        \item[3.1] 実験方法\\
        回路図を以下に示す.
        \begin{figure}[H]
        \centering
        \includegraphics[width=9cm]{/home/asakayusuke/Pictures/z32.png}
        \caption{オペアンプを使ったコンパレータ回路 (引用元:\cite{Okumura})}
        \label{figure:i}
        \end{figure}
        \begin{figure}[H]
            \centering
            \includegraphics[width=6cm]{/home/asakayusuke/Pictures/z33.png}
            \caption{オペアンプへのコンデンサの接続方法 (引用元:\cite{Okumura})}
            \label{figure:cc}
            \end{figure}
        図\ref{figure:i}のようにコンパレータ回路を作成する. しかし, オペアンプの+V\textsubscript{CC}, -V\textsubscript{CC}
        接続端子とその対応する電源との間には必ずコンデンサ (10\(\mu F\))を電源の向きにして付ける. また, 可変抵抗も
        向きに気をつけよう (以下の図参照). オペアンプの非反転入力端子 (+側)に付ける直流電源は, 1.5[V]乾電池4本の箱を使う. 
        \\
        \begin{figure}[H]
        \centering
        \includegraphics[width=4cm]{/home/asakayusuke/Pictures/z35.png}
        \caption{可変抵抗の接続向き}
        \label{figure:teiko}
        \end{figure}
        \\
        回路ができたら入力電圧V\textsubscript{i}を\(-6[V]から6[V]\)まで増加させながら, 出力電圧を測定する. その際は
        デジタルマルチメーターで電圧をできるだけ整数に近い値になるように測定しながら可変抵抗を回すようにする. また, 0[V]
        付近だと変化が不連続なので細かく変化させて測定する必要がある. 測定したら非反転入力端子を交流電源 (発振器)に変えて, 
        オシロスコープのプローブのch.1で入力端子の, ch.2で出力端子の波形を観察する. そのとき, 周波数は100[Hz], 振幅電圧は5[V]とする. \\

        \item[3.2] 使用機器\\
        直流電源 (AD-8723D), 直流電流計 (DCZA21), ユニバーサルボード (TOKUSHUSEIDO, BD 39),デジタルマルチメーター (東陽テクニカ製), 
        オペアンプ (UA741CP), 発振器\\

        \item[3.3] 実験結果\\
        実験1の結果として表\ref{table:gy}を以下に示す.
        \begin{table}[H]
            \centering
            \caption{コンパレータ回路の入出力特性}
            \label{table:gy}
            \includegraphics[width=4cm]{/home/asakayusuke/Pictures/z36.png}
        \end{table}
        \\
        表\ref{table:gy}を参照して作成したグラフが図\ref{figure:to}\\
        \begin{figure}[H]
            \centering
            \includegraphics[width=9cm]{/home/asakayusuke/Pictures/z37.png}
            \caption{コンパレータ回路の入出力特性のグラフ}
            \label{figure:to}
        \end{figure}
        このグラフでは, 入力電圧が0[V]のときを境にして\(-\) (負)のとき\(-0.7[V]付近\), + (正)のとき7[V]付近に出力電圧が
        上下している. しかし, 原理より,コンパレータ回路の特性は電源電圧V\textsubscript{CC}の辺りまで上下する性質を持つので
        あった. なので, これは考察の部分で取り上げることにする. \\

        そして, 交流電圧に切り替えて記録した波形である. 
        \begin{figure}[H]
            \centering
            \includegraphics[width=9cm]{/home/asakayusuke/Pictures/IMG_0106.PNG}
            \caption{コンパレータ回路の入出力電圧の波形}
        \end{figure}

        \item[3.4] 考察 (失敗原因の考察)\\
        今回の実験1ではコンパレータ回路を使用した. しかしコンパレータ回路の性質に合わない結果の図\ref{figure:to}になってしまった. 
        その原因を考察しようと思う. \\

        1つとして, 実験の回路が最初に電源に繋いだときに間違っていた可能性がある. なぜなら, この後電源電圧がおかしい可能性を
        探って回したとき, オペアンプが故障してしまったからである. 筆者は回路の間違えていた箇所は可変抵抗の場所と推測した. 
        可変抵抗の接続向きが今回使用したものが一般的な可変抵抗と違ったため, このような結果になったと考えられる. \\
        
        \item[3.5] 検討\\
        コンパレータ回路に直流電圧を入力したときに得られた入出力特性のグラフの特徴をオペアンプの性質を使って説明\\
        今回の実験結果では性質通りのグラフにはならなかったが, 上手く実験をすると以下のようになる. 
        \begin{figure}[H]
            \centering
            \includegraphics[width=7cm]{/home/asakayusuke/Pictures/z66.png}
            \caption{コンパレータ回路の性質}
        \end{figure}
        \\
        上図の通りになった入出力特性は, まず反転入力端子が接地しており,これで\(V_-=0[V]\)が基準電圧となる. そして, 
        非反転入力端子の入力電圧\(V_+\), つまり入力電圧\(V_i\)が\(V_i>V_-\)のとき, 出力電圧\(V_o\)は+V\textsubscript{CC}となり, 
        入力電圧\(V_i\)が\(V_i<V_-\)のとき, 出力電圧\(V_o\)は\(-V\textsubscript{CC}\)となる. \\

        コンパレータ回路に正弦波の交流電圧を入力したときに得られた出力波形の特徴についてオペアンプの性質を使って説明\\
        
        正弦波の入力電圧が加えられた場合, 今回の実験では基準電圧が0[V]である. \(V_i\)が一定ではなく正負が交互に訪れるため, 
        +V\textsubscript{CC}と\(-V\textsubscript{CC}\)が半周期ごとに入れ替わることになる. \\

        図2で非反転入力端子を基準電圧として反転入力端子に入力電圧Viを入力したとする。入力電圧Viを直流電圧としたときの入出力特性の
        グラフはどう変化すると考えられるか?また入力電圧を交流電圧としたときの入出力波形の関係はどう変化すると考えられるか?
        \\

        どちらも, 今回の実験では負の値から出力されているグラフが, 正の値から始まるようになる. \\

    \end{enumerate}
    \item {\bf \large 【実験2】ボルテージ・フォロワ回路}\\
    \begin{enumerate}
        \item[4.1]実験方法\\
        \begin{figure}[H]
            \centering
            \includegraphics[width=9cm]{/home/asakayusuke/Pictures/z34.png}
            \caption{ボルテージ・フォロワ回路 (引用元:\cite{Okumura})}
            \label{figure:gq}
        \end{figure}
        上の図\ref{figure:gq}を参照しながらオペアンプの負帰還回路にしたボルテージ・フォロワ回路を作成する. オペアンプへ
        供給する直流電圧V\textsubscript{CC}を12Vに調整する. そしたら, 可変抵抗を回すことで-6[V]\(\sim \) 6[V]まで増加させながら, 
        入力電圧\(V_i\)と出力電圧\(V_o\)を測定する. それと同時に非反転入力端子と反転入力端子の間の電圧\(V_+-V_-\)も測定する. 
        測定したら非反転入力端子を交流電源 (発振器)に変えて, オシロスコープのプローブのch.1で入力端子の, ch.2で出力端子の
        波形を観察する. そのとき, 周波数は100[Hz], 振幅電圧は5[V]とする. \\

        \item[4.2]使用機器\\
        直流電源 (AD-8723D), 直流電流計 (DCZA21), ユニバーサルボード (TOKUSHUSEIDO, BD 39),デジタルマルチメーター (東陽テクニカ製), 
        オペアンプ (UA741CP), 発振器\\

        \item[4.3] 実験結果\\
        実験1の結果として表\ref{table:gy}を以下に示す.
        \begin{table}[H]
            \centering
            \caption{ボルテージ・フォロワ回路の入出力特性}
            \label{table:g2}
            \includegraphics[width=4cm]{/home/asakayusuke/Pictures/z38.png}
        \end{table}
        \\
        表\ref{table:g2}を参照して作成したグラフが図\ref{figure:io}\\
        \begin{figure}[H]
            \centering
            \includegraphics[width=9cm]{/home/asakayusuke/Pictures/z39.png}
            \caption{ボルテージ・フォロワ回路の入出力特性のグラフ}
            \label{figure:io}
        \end{figure}
        このグラフから, 入力電圧と出力電圧の関係は, 補足説明の式 (\ref{v5})のようになることがわかる.\\

        そして, 交流電圧に切り替えて記録した波形である
        \begin{figure}[H]
            \centering
            \includegraphics[width=9cm]{/home/asakayusuke/Pictures/IMG_3918.PNG}
            \caption{ボルテージ・フォロワ回路の入出力電圧の波形}
        \end{figure}

        \item[4.4] 考察・検討\\
        【実験 1】で調べたコンパレータ回路と【実験 2】で調べたボルテージ・フォロワ回
        路の入出力特性にはどのような違いが見られたかを説明し、その理由を検討せよ。\\

        実験1のコンパレータ回路では, 入力電圧が0[V]になるのを境にして出力電圧が急激に+V\textsubscript{CC}, \(-V\textsubscript{CC}\)と
        上下していたが, 実験2のボルテージ・フォロワ回路では入力電圧と出力電圧が1対1の関係になっている. これは, 実験2で出力部から
        負帰還をしているため, 増幅率が1になるからである. \\

        【実験 1】のコンパレータ回路では反転入力端子(-端子)が接地されていたため、非
        反転入力端子(+端子)と反転入力端子(-端子)の間の電圧(V + -V - ) は入力電圧 V i そのもの
        であった。しかしながら負帰還がかかったボルテージ・フォロワ回路では電圧(V + -V - )は
        ほぼゼロであった。つまり+端子と-端子はほぼ短絡しているとみなせる。なぜ負帰還が
        かかったオペアンプ回路では電圧(V + -V - )はほぼゼロとなるのか?各自文献などで調べて説明せよ。\\

        オペアンプを使った反転増幅回路を考える. 以下の文の参考:\cite{kaso}
        \begin{figure}[H]
            \centering
            \includegraphics[width=6cm]{/home/asakayusuke/Pictures/z98.PNG}
            \caption{仮想短絡させた差動増幅回路}
            \label{hg}
        \end{figure}
        この図\ref{hg}から, オペアンプの増幅率を\(A_v\)とすると, 帰還しない場合, cに\(V_o=-A_v V_i\)が出力される. しかし, この
        回路には\(R_1\)による帰還があることでaにはcからの\(V_o\)が加わる. よってaの電圧が下がっていく. この現象が連続するためaの電圧
        は徐々に下がっていく. やがてaの電圧がアースに対して負になると, cには正の電圧が出力される. すると逆にaの電圧が高くなる. この動作
        は増幅率の高いオペアンプで瞬時に起こるので, 結局aの電圧は0になる. よって, オペアンプの実際の入力インピーダンスは\(\infty \) と
        なるのにaとbはあたかもショートしているように見える. これを仮想短絡という. 
        この仮想短絡と原理の補足説明を使って, 式 (\ref{v5})より
        \begin{equation}
            \(
            V_o=V_i
            \)
        \end{equation}
        となる. \\
    \end{enumerate}
    
    \item {\bf \large 結論}\\
    \quad 目的としていたオペアンプ単体で使用した時の基本的な性質について理解できた. また, オペアンプ回路の実際の使用例である
    ボルテージ・フォロワ回路を使って負帰還回路の性質を学べた.\\

    \begin{thebibliography}{9}
    \bibitem{suken} [104 数研 物理 313]改訂版 物理, 数研出版, (2023.4,23)
    \bibitem{Okumura} 政田元太, インテリジェントデバイス実験授業資料「オペアンプ基本特性」, 玉川大学工学部情報通信工学科, 
    pp.1\(\sim \)11 (2023.4,19)
    \bibitem{kaso} 堀桂太郎, 基礎マスターシリーズ オペアンプの基礎マスター, 電気書院, pp.43,44,55,181,182 (2023.5,16)
    \end{thebibliography}
\end{enumerate}
\end{document}
